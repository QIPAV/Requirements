\section{Scrum}
Scrum is a iterative and incremental development framework that coincides with the agile philosophy. Scrum handles requirements a little different than many other more traditional project models. In Scrum the Prouct Backlog is the backbone of the project and it is our equivalent to a requirements document. A short introduction to some important elements in Scrum will follow.  


\subsection{Backlog}
Scrum has two types of backlogs, the Product Backlog and the Sprint Backlog.
\\ \\ 
\textbf{The Product Backlog} is a prioritized list of customer-centric features. It breaks the project and every thing you wish to do down into managable increments of work called Product Backlog Items. These are what we call "User Stories". Product backlog items are not tasks, but can be broken down into smaller tasks. Items toward the top of the Product Backlog should be small enough that a team could accomplish several of them (including proper testing, integration, etc.) in two-week Sprint. The Product Backlog are prioritized from highest- to lowest priority where the topmost item is the one with the highest priority.   
\\ \\ 
\textbf{The Sprint Backlog} is a collection of the user stories which the team plan to accomplish during the Sprint. This is normally created at the Sprint Planning meeting where team members can pick stories or task from the Product Backlog and drag it into the Sprint Backlog. The team decide for themselves what they are able to do in the next Sprint. After the team has selected User Stories, they are broken down to smaller tasks. These tasks are sorted as "to do", "in progress", "done" and "approved". As a general rule, one task should never exeed 3 days of work and should ideally be less than one days work. If they are bigger than this, they should be broken down further to smaller more managaeable tasks. If Product Backlog Items are estimated, they’re estimated in relative units such as story points.
\\ \\ 
To summarize, the Product Backlog concerns everything the team might ever do, or the "requirements", and the Sprint Backlog concern the User Stories which is intended to be accomplished during the next Sprint.  
  
%%%%%%%%


\subsection{User Stories}
The Product Backlog Items (PBIs) are often referred to as User Stories. These are features or "requirements" deduced from what the customer wants. These are often written from the perspective of the end user and with the customer in mind. The syntax is \textit{As a/an(user), I want/need, So that(purpose)}
\\ \\
\textbf{Example of a User Story} \\
As an customer, I want the quadrotor to weigh less than 2.5kg. So that laws and regulations are held 

\subsection{Acceptance Criteria}
In Scrum each user story has a Acceptance Criterion. It provide the requirement thet have to be met for a user story to be assessed as complete.  
The syntax of these are: \textit{Given(some context), When(some action is carried out), Then(something expected).} \\
\\ 
\textbf{An example an Acceptance Criterion} \\
Given that we have a quadrotor, When the quadrotor is weighed Then it shall be less than 2.5kg.



