\section{Scrum}
We are using Scrum as the Project Model for our project. Scrum handles requirements a little differently than other more traditional project models.  


Our requirement document are essentially called Product Backlog in Scrum. This backlog concerns on every thing we might ever do in the project. 
The backlog are prioritized and order from highest- to lowest priority.


\subsection{Backlog}

Scrum has two types of backlogs, the Product Backlog, and the Sprint Backlog.


%%%%%%%%
The Product Backlog is a prioritized list of customer-centric features. It breaks the big-picture vision down into manageable increments of work called Product Backlog Items (PBIs). These are typically expressed in user story form. Product Backlog Items are not tasks, they represent the what rather than the how. If Product Backlog Items are estimated, they’re estimated in relative units such as story points. Items toward the top of the Product Backlog should be small enough that a team could accomplish several of them (including proper testing, integration, etc.) in one two-week Sprint.
%%%%%%%%


\subsection{User Stories}
Scrum use "User Stories" as requirements. These user stories are placed in the Product Backlog and prioritized. 

Backlog items are selected to go into the sprint backlog, and are then broken down by the team into tasks. Tasks are sorted as "to do", "in progress" and "done". The team are free to take on the tasks they see fit and can manage. As a general rule, one task should never exceed 3 days and should preferably be less than one days work. If tasks are bigger than this, they should be broken into smaller more manageable pieces.

The syntax is As a/an, I want/need, So that(purpose) 
An example of a User Story: As an engineer, I want the quadrotor to weigh less than 2.5kg. So that laws and regulations are held 

Written from the perspective of the end user

\subsection{Acceptance Criteria}
In Scrum each user story has a Acceptance Criterion. The syntax of these Acceptance Criteria are: GIVEN, WHEN, THEN. An example of this is Given that we have a quadrotor, when the quadrotor is weighed then it shall be less than 2.5kg.


 Acceptance criteria are the requirements that have to be met for a story to be assessed as complete.
