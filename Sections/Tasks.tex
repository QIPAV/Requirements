\section{Tasks}
During the Sprint Planning Meeting when user stories are put into the Sprint Backlog, the team also breaks the user stories down to even smaller \textit{tasks}. These tasks are then rated with Story Points relative to each other. These Story Point is a estimate of the work neccessary to complete the task. Since we are using Jira to monitor all of the tasks we had to write them down in a table for you to see. Below follows a collection of all our tasks and their status during our project.


%                   SPRINT 1                       %
%%%%%%%%%%%%%%%%%%%%%%%%%%%%%%%%%%%%%%%%%%%%%%%%%%%%
\centering\LARGE\textbf{Sprint 1}
\begin{table}[ht]
\begin{tabularx}{\linewidth}{|m{1.85cm} m{5.5cm} m{1.2cm} m{1.5cm} m{4.3cm}|}
\hline
    \rowcolor{cadetgrey} 
     \textbf{Jira ID:} & \textbf{Description:} & \textbf{BL ID} & \textbf{Story Points:} & \textbf{Status: } \\ \hline
      VPQ-61 & Get arduino Car & BL001 & 3\centering & Moved to Sprint 2  \\ 
     \rowcolor{gainsboro} x & x & x & x\centering & x  \\
      x & x & x & x\centering & x  \\
      \rowcolor{gainsboro}x & x & x & x\centering & x  \\
      x & x & x & x\centering & x  \\
     \rowcolor{gainsboro} x & x & x & x\centering & x  \\
      x & x & x & x\centering & x  \\
     \rowcolor{gainsboro} x & x & x & x\centering & x  \\    
      x & x & x & x\centering & x  \\
    \rowcolor{gainsboro}  x & x & x & x\centering & x  \\
\hline    
\end{tabularx}
\caption{Sprint 1}
\end{table}
%%%%%%%%%%%%%%%%%%%%%%%%%%%%%%%%%%%%%%%%%%%%%%%%%%%%%%


\begin{comment}
%                   SPRINT #                       %
%%%%%%%%%%%%%%%%%%%%%%%%%%%%%%%%%%%%%%%%%%%%%%%%%%%%
\centering\LARGE\textbf{Sprint #}
\begin{table}[ht]
\begin{tabularx}{\linewidth}{|m{1.85cm} m{5.5cm} m{1.2cm} m{1.5cm} m{4.3cm}|}
\hline
    \rowcolor{cadetgrey} 
     \textbf{Jira ID:} & \textbf{Description:} & \textbf{BL ID} & \textbf{Story Points:} & \textbf{Status: } \\ \hline
      x & x & x & x\centering & x  \\ 
\rowcolor{gainsboro} x & x & x & x\centering & x  \\
      x & x & x & x\centering & x  \\
\rowcolor{gainsboro}x & x & x & x\centering & x  \\
      x & x & x & x\centering & x  \\
\rowcolor{gainsboro} x & x & x & x\centering & x  \\
      x & x & x & x\centering & x  \\
\rowcolor{gainsboro} x & x & x & x\centering & x  \\    
      x & x & x & x\centering & x  \\
\rowcolor{gainsboro}  x & x & x & x\centering & x  \\
\hline    
\end{tabularx}
\caption{Sprint #}
\end{table}
%%%%%%%%%%%%%%%%%%%%%%%%%%%%%%%%%%%%%%%%%%%%%%%%%%%%%%
\end{comment}